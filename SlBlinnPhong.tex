\begin{frame}
\frametitle{Blinn-Phong: Demo}
\begin{figure}[ht]
    \centering
    \includegraphics[scale=0.25]{images/SlidesBlinnPhong/SceneMaterialsLight.png}
\end{figure}
\end{frame}

\begin{frame}
\frametitle{Blinn-Phong}
\begin{itemize}
\item L'illuminazione viene divisa in tre componenti: ambient, diffuse e specular
\item La componente ambientale modella il fatto che una scena non è mai totalmente buia
\item La componente diffusiva simula l'impatto che la luce ha sugli oggetti opachi
\item La componente speculare simula il punto luminoso che una luce causa su oggetti lucidi (specular highlight)
\item Ogni oggetto che deve essere illuminato ha un materiale: componente ambientale, diffusiva e speculare
\item Un materiale ha una proprietà aggiuntiva che indica quanto un oggetto è lucido
\item Una luce ha diverse intensità per la componente ambientale, diffusiva e speculare
\end{itemize}
\end{frame}
